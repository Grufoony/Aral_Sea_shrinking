The general technique to satellite image processing presents some difficulties with acquired images, so, in order to proceed with it, several steps has been made in order to correct our images and  make them suitable for our purpose.  

Acquired images can be generated by different satellites or different lenses and they could have a wide range of extension and resolution. 
We need many images to cover a long-time interval, which may be very different among themselves, so we need to preprocess them in order to overcome those problems. 

The first step is to convert all images in the way they have all the same resolution, with a fixed tolerance. 
In our case we chose photos from a dataset were already re-sampled. 

We must be sure that all images describe the same geographic area. This is essential to compare them, but also to overlap them for both qualitative and quantitative analysis. 
In the study of a big area, it’s common to have also geometric distortion. We have used images already geolocalized which are referred to the same area. 
Landsat’s dataset has a defined path/row extension. Row refers to the latitudinal center line of a frame of imagery. As the satellite moves along its path, the observatory instruments are continuously scanning the terrain below. 
We can convert the path/row of the satellite to a value of lat/long coordinates.  

Our analyzed images were taken from a dataset and for our analysis geometric distortion is not important. 

In order to remove clouds, the general technique concerns to blend multi-spectral images and use a reference no-cloudy image. 
Some algorithms can be improved to distinguish clouds from Earth’s surface. Our images were already cloudless since the Aral Sea is situated in a desert area and clouds are particularly rare in its place. 

For some research issues, we need multi-spectral images \cite{satelliteImg}: sometimes using different material reflectance is the only way to segmentate different objects. For our purposes, it wasn’t essential to use multi-spectral images because the differences between the lake and the surrounding area were evident in terms of both colors and grey levels, so for this work we just used true color images.  
