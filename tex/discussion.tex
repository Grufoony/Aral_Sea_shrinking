Image processing has revealed as an excellent instrument applied to satellite images. 
In particular, we were able to quantify the shrinking ratio of the Aral Sea during a 44-years timespan by using a combination of techniques, among which k-means clustering and threshold. 
The result is an almost linear decay, explainable qualitatively by a combination of precipitation/evaporation phenomena.

Furthermore, we were also able to improve a time-evolution visualization by using an edge detection technique and creating an animated GIF.
Looking at pictures in Fig. \ref{fig:appendixedges} we can see how the area of the northern lake has remained almost constant during the years, due to the preservation attempts that were made. 

Observations gathered by multiple earth observation satellites, such as Landsat, combined with the processing techniques serves as a common, reliable record for environmental change around the world. 
Indeed, in the last half century, the record of earth observation from space has become the indispensable foundation of almost all deliberations about the state of the planet only to study our case, but also to keep track on all kinds of environmental changes which can put us a step ahead of any environmental catastrophe.
